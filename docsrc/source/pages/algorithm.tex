\documentclass[utf8]{ctexart}

\usepackage[ruled]{algorithm2e}    

\begin{document}
\begin{algorithm}[H]
    \caption{形成第i层的节点集}
    \KwIn{图G}
    \KwOut{某层的节点集}

    为图中所有种类(包括Scope)的节点分配独立ID\;
    按种类进行分类,获得Scope节点集$S_{scope}$,算子节点集$S_{normal}$,参数和常数节点集$S_{parameter}$\;
    确定所有节点的层级\;

    \uIf {i==1} {
        输出$S_{normal}$+$S_{parameter}$\;
    } \Else {
        筛选$S_{scope}$,获得层级i的节点集$S_{scope-i}$\;
        \For{$S_{scope-i}$中的节点$n$}{
            重定向$n$的上下游节点id至层级i\;
        }
        输出$S_{scope-i}$\;
    }

\end{algorithm}

\begin{algorithm}[H]
    \caption{子图挖掘}
    \KwIn{节点集$S_{node}$}
    \KwOut{子图集的集合$S_{subgraph}$}

    统计节点集$S_{node}$中的频繁节点\;
    将频繁节点标记为单节点的核,放入计算池中\;

    \While{计算池不为空}{
        从计算池中取出一个核$C_{i}$\;
        \For{Ci的子图模式中的边界点$P_{ij}$}{
            \For{Ci的所有子图实例中Pij位置上的对应节点$N_{ijk}$}{
                记录节点$N_{ijk}$的所有下游节点的种类\;
            }
            统计节点$N_{ijk}$的下游节点的种类频次\;
            \For{节点$N_{ijk}$的下游节点的种类$T_{ijkl}$}{
                \If{生长后的新核$C_{ijkl}$满足各种阈值条件,且未被注册}
                {
                    注册该新核$C_{ijkl}$\;
                    将新核放入计算池\;
                }
            }
        }\If{无法生长出新核,且$C_{i}$是有效子图集}
        {
            将核$C_{i}$的拷贝放入子图集的集合$S_{subgraph}$中\;
        }
        删除$C_{i}$\;
        \For{$S_{subgraph}$中的子图$SG_i$}{
            \For{$S_{subgraph}$中除$SG_i$的其他子图$SG_j$}{
                \If{$SG_i$是$SG_j$的子图,且$SG_i$的各项参数不符合带罚项的阈值条件}
                {
                    移除$SG_i$\;
                }
            }
        }
    }
    输出$SG_i$\;

\end{algorithm}
\end{document}